% Options for packages loaded elsewhere
\PassOptionsToPackage{unicode}{hyperref}
\PassOptionsToPackage{hyphens}{url}
%
\documentclass[
]{article}
\usepackage{amsmath,amssymb}
\usepackage{iftex}
\ifPDFTeX
  \usepackage[T1]{fontenc}
  \usepackage[utf8]{inputenc}
  \usepackage{textcomp} % provide euro and other symbols
\else % if luatex or xetex
  \usepackage{unicode-math} % this also loads fontspec
  \defaultfontfeatures{Scale=MatchLowercase}
  \defaultfontfeatures[\rmfamily]{Ligatures=TeX,Scale=1}
\fi
\usepackage{lmodern}
\ifPDFTeX\else
  % xetex/luatex font selection
\fi
% Use upquote if available, for straight quotes in verbatim environments
\IfFileExists{upquote.sty}{\usepackage{upquote}}{}
\IfFileExists{microtype.sty}{% use microtype if available
  \usepackage[]{microtype}
  \UseMicrotypeSet[protrusion]{basicmath} % disable protrusion for tt fonts
}{}
\makeatletter
\@ifundefined{KOMAClassName}{% if non-KOMA class
  \IfFileExists{parskip.sty}{%
    \usepackage{parskip}
  }{% else
    \setlength{\parindent}{0pt}
    \setlength{\parskip}{6pt plus 2pt minus 1pt}}
}{% if KOMA class
  \KOMAoptions{parskip=half}}
\makeatother
\usepackage{xcolor}
\usepackage[margin=1in]{geometry}
\usepackage{color}
\usepackage{fancyvrb}
\newcommand{\VerbBar}{|}
\newcommand{\VERB}{\Verb[commandchars=\\\{\}]}
\DefineVerbatimEnvironment{Highlighting}{Verbatim}{commandchars=\\\{\}}
% Add ',fontsize=\small' for more characters per line
\usepackage{framed}
\definecolor{shadecolor}{RGB}{248,248,248}
\newenvironment{Shaded}{\begin{snugshade}}{\end{snugshade}}
\newcommand{\AlertTok}[1]{\textcolor[rgb]{0.94,0.16,0.16}{#1}}
\newcommand{\AnnotationTok}[1]{\textcolor[rgb]{0.56,0.35,0.01}{\textbf{\textit{#1}}}}
\newcommand{\AttributeTok}[1]{\textcolor[rgb]{0.13,0.29,0.53}{#1}}
\newcommand{\BaseNTok}[1]{\textcolor[rgb]{0.00,0.00,0.81}{#1}}
\newcommand{\BuiltInTok}[1]{#1}
\newcommand{\CharTok}[1]{\textcolor[rgb]{0.31,0.60,0.02}{#1}}
\newcommand{\CommentTok}[1]{\textcolor[rgb]{0.56,0.35,0.01}{\textit{#1}}}
\newcommand{\CommentVarTok}[1]{\textcolor[rgb]{0.56,0.35,0.01}{\textbf{\textit{#1}}}}
\newcommand{\ConstantTok}[1]{\textcolor[rgb]{0.56,0.35,0.01}{#1}}
\newcommand{\ControlFlowTok}[1]{\textcolor[rgb]{0.13,0.29,0.53}{\textbf{#1}}}
\newcommand{\DataTypeTok}[1]{\textcolor[rgb]{0.13,0.29,0.53}{#1}}
\newcommand{\DecValTok}[1]{\textcolor[rgb]{0.00,0.00,0.81}{#1}}
\newcommand{\DocumentationTok}[1]{\textcolor[rgb]{0.56,0.35,0.01}{\textbf{\textit{#1}}}}
\newcommand{\ErrorTok}[1]{\textcolor[rgb]{0.64,0.00,0.00}{\textbf{#1}}}
\newcommand{\ExtensionTok}[1]{#1}
\newcommand{\FloatTok}[1]{\textcolor[rgb]{0.00,0.00,0.81}{#1}}
\newcommand{\FunctionTok}[1]{\textcolor[rgb]{0.13,0.29,0.53}{\textbf{#1}}}
\newcommand{\ImportTok}[1]{#1}
\newcommand{\InformationTok}[1]{\textcolor[rgb]{0.56,0.35,0.01}{\textbf{\textit{#1}}}}
\newcommand{\KeywordTok}[1]{\textcolor[rgb]{0.13,0.29,0.53}{\textbf{#1}}}
\newcommand{\NormalTok}[1]{#1}
\newcommand{\OperatorTok}[1]{\textcolor[rgb]{0.81,0.36,0.00}{\textbf{#1}}}
\newcommand{\OtherTok}[1]{\textcolor[rgb]{0.56,0.35,0.01}{#1}}
\newcommand{\PreprocessorTok}[1]{\textcolor[rgb]{0.56,0.35,0.01}{\textit{#1}}}
\newcommand{\RegionMarkerTok}[1]{#1}
\newcommand{\SpecialCharTok}[1]{\textcolor[rgb]{0.81,0.36,0.00}{\textbf{#1}}}
\newcommand{\SpecialStringTok}[1]{\textcolor[rgb]{0.31,0.60,0.02}{#1}}
\newcommand{\StringTok}[1]{\textcolor[rgb]{0.31,0.60,0.02}{#1}}
\newcommand{\VariableTok}[1]{\textcolor[rgb]{0.00,0.00,0.00}{#1}}
\newcommand{\VerbatimStringTok}[1]{\textcolor[rgb]{0.31,0.60,0.02}{#1}}
\newcommand{\WarningTok}[1]{\textcolor[rgb]{0.56,0.35,0.01}{\textbf{\textit{#1}}}}
\usepackage{graphicx}
\makeatletter
\def\maxwidth{\ifdim\Gin@nat@width>\linewidth\linewidth\else\Gin@nat@width\fi}
\def\maxheight{\ifdim\Gin@nat@height>\textheight\textheight\else\Gin@nat@height\fi}
\makeatother
% Scale images if necessary, so that they will not overflow the page
% margins by default, and it is still possible to overwrite the defaults
% using explicit options in \includegraphics[width, height, ...]{}
\setkeys{Gin}{width=\maxwidth,height=\maxheight,keepaspectratio}
% Set default figure placement to htbp
\makeatletter
\def\fps@figure{htbp}
\makeatother
\setlength{\emergencystretch}{3em} % prevent overfull lines
\providecommand{\tightlist}{%
  \setlength{\itemsep}{0pt}\setlength{\parskip}{0pt}}
\setcounter{secnumdepth}{-\maxdimen} % remove section numbering
\ifLuaTeX
  \usepackage{selnolig}  % disable illegal ligatures
\fi
\IfFileExists{bookmark.sty}{\usepackage{bookmark}}{\usepackage{hyperref}}
\IfFileExists{xurl.sty}{\usepackage{xurl}}{} % add URL line breaks if available
\urlstyle{same}
\hypersetup{
  pdftitle={Density estimation. GMM. DBSCAN (Assignment)},
  pdfauthor={Louis Van Langendonck \& Antonin Rosa \& Albert Martín},
  hidelinks,
  pdfcreator={LaTeX via pandoc}}

\title{Density estimation. GMM. DBSCAN (Assignment)}
\author{Louis Van Langendonck \& Antonin Rosa \& Albert Martín}
\date{11/oct./2023}

\begin{document}
\maketitle

\begin{Shaded}
\begin{Highlighting}[]
\FunctionTok{load}\NormalTok{(}\StringTok{"BikeDay.Rdata"}\NormalTok{)}
\NormalTok{X }\OtherTok{\textless{}{-}} \FunctionTok{as.matrix}\NormalTok{(day[day}\SpecialCharTok{$}\NormalTok{yr}\SpecialCharTok{==}\DecValTok{1}\NormalTok{,}\FunctionTok{c}\NormalTok{(}\DecValTok{10}\NormalTok{,}\DecValTok{14}\NormalTok{)])}
\CommentTok{\#pairs(X)}
\end{Highlighting}
\end{Shaded}

\hypertarget{model-based-clustering-assuming-a-gaussian-mixture-model}{%
\subsection{1. model based clustering assuming a Gaussian Mixture
Model}\label{model-based-clustering-assuming-a-gaussian-mixture-model}}

Using the G parameter in the \texttt{Mclust} function, options for K are
set between 2 and 6. It automatically selects the model with the highest
BIC score. Moreover, the BIC plot Via the \texttt{modelNames} parameter,
mixture components are allowed variation in volume, shape, and
orientation. The `BIC'-plot shows that K = 3 scores the best and is thus
the choice of K the model automatically continues with. The
classification, uncertainty and density plots use this value of K

\begin{Shaded}
\begin{Highlighting}[]
\FunctionTok{library}\NormalTok{(mclust)}
\NormalTok{GMM }\OtherTok{\textless{}{-}} \FunctionTok{Mclust}\NormalTok{(X,}\AttributeTok{G=}\DecValTok{2}\SpecialCharTok{:}\DecValTok{6}\NormalTok{,}\AttributeTok{modelNames=}\StringTok{"VVV"}\NormalTok{)}
\FunctionTok{plot}\NormalTok{(GMM, }\StringTok{"BIC"}\NormalTok{)}
\end{Highlighting}
\end{Shaded}

\includegraphics[height=0.4\textheight]{LouisVanLangendonck_AntoninRosa_files/figure-latex/unnamed-chunk-2-1}

\begin{Shaded}
\begin{Highlighting}[]
\FunctionTok{plot}\NormalTok{(GMM, }\StringTok{"classification"}\NormalTok{)}
\end{Highlighting}
\end{Shaded}

\includegraphics[height=0.4\textheight]{LouisVanLangendonck_AntoninRosa_files/figure-latex/unnamed-chunk-2-2}

\begin{Shaded}
\begin{Highlighting}[]
\FunctionTok{plot}\NormalTok{(GMM, }\StringTok{"uncertainty"}\NormalTok{)}
\end{Highlighting}
\end{Shaded}

\includegraphics[height=0.4\textheight]{LouisVanLangendonck_AntoninRosa_files/figure-latex/unnamed-chunk-2-3}

\begin{Shaded}
\begin{Highlighting}[]
\FunctionTok{plot}\NormalTok{(GMM, }\StringTok{"density"}\NormalTok{)}
\end{Highlighting}
\end{Shaded}

\includegraphics[height=0.4\textheight]{LouisVanLangendonck_AntoninRosa_files/figure-latex/unnamed-chunk-2-4}

\hypertarget{compare-the-previous-density-plot-with-the-non-parametric-density-estimation-of-tempcasual.}{%
\subsection{2. Compare the previous density plot with the non-parametric
density estimation of
(temp,casual).}\label{compare-the-previous-density-plot-with-the-non-parametric-density-estimation-of-tempcasual.}}

Comparing these plots, it becomes clear both show a similar profile.
However the GMM model (left) yields a clear distinction in three
gaussian shapes (as typically predicted in this type of model), while
the non-param model (right) shows more arbitrary outlines. This is due
to the fact that in the first case, the data is fit to the model
(mixture of three gaussians) and the plot is a visual representation of
these three gaussians, while the second plot is a direct representation
of the data density, and thus less regular.

\begin{Shaded}
\begin{Highlighting}[]
\FunctionTok{library}\NormalTok{(sm)}
\FunctionTok{par}\NormalTok{(}\AttributeTok{mfrow =} \FunctionTok{c}\NormalTok{(}\DecValTok{1}\NormalTok{, }\DecValTok{2}\NormalTok{))}
\FunctionTok{plot}\NormalTok{(GMM,}\StringTok{"density"}\NormalTok{,}\AttributeTok{main=}\StringTok{"GMM"}\NormalTok{)}
\NormalTok{a }\OtherTok{\textless{}{-}} \FloatTok{0.25}
\NormalTok{h\_prop }\OtherTok{\textless{}{-}}\NormalTok{ a}\SpecialCharTok{*}\FunctionTok{c}\NormalTok{(}\FunctionTok{sd}\NormalTok{(X[,}\StringTok{"temp"}\NormalTok{]), }\FunctionTok{sd}\NormalTok{(X[,}\StringTok{"casual"}\NormalTok{]))}
\FunctionTok{sm.density}\NormalTok{(X,h\_prop,}\AttributeTok{display=}\StringTok{"slice"}\NormalTok{,}\AttributeTok{main=}\StringTok{"Non{-}param Denisty Estimator"}\NormalTok{)}
\end{Highlighting}
\end{Shaded}

\includegraphics{LouisVanLangendonck_AntoninRosa_files/figure-latex/unnamed-chunk-3-1.pdf}

\hypertarget{non-parametric-joint-density-estimation-of-each-of-the-clusters}{%
\subsection{3. Non-parametric joint density estimation of each of the
clusters}\label{non-parametric-joint-density-estimation-of-each-of-the-clusters}}

\begin{Shaded}
\begin{Highlighting}[]
\NormalTok{clust.ind }\OtherTok{\textless{}{-}}\NormalTok{ GMM}\SpecialCharTok{$}\NormalTok{classification}
\FunctionTok{plot}\NormalTok{(X,}\AttributeTok{col=}\NormalTok{clust.ind)}
\ControlFlowTok{for}\NormalTok{ (j }\ControlFlowTok{in} \DecValTok{1}\SpecialCharTok{:}\DecValTok{3}\NormalTok{)\{}
\NormalTok{  cl.j }\OtherTok{\textless{}{-}}\NormalTok{ (clust.ind}\SpecialCharTok{==}\NormalTok{j)}
  \FunctionTok{sm.density}\NormalTok{(}
\NormalTok{    X[cl.j,],}
    \AttributeTok{h=}\FloatTok{0.4}\SpecialCharTok{*}\FunctionTok{c}\NormalTok{(}\FunctionTok{sd}\NormalTok{(X[cl.j,}\StringTok{"temp"}\NormalTok{]),}\FunctionTok{sd}\NormalTok{(X[cl.j,}\StringTok{"casual"}\NormalTok{])),}
    \AttributeTok{display=}\StringTok{"slice"}\NormalTok{,}
    \AttributeTok{props=}\FunctionTok{c}\NormalTok{(}\DecValTok{75}\NormalTok{),}
    \AttributeTok{col=}\NormalTok{j, }
    \AttributeTok{cex=}\DecValTok{4}\NormalTok{, }
    \AttributeTok{add=}\ConstantTok{TRUE}\NormalTok{)}
\NormalTok{\}}
\end{Highlighting}
\end{Shaded}

\includegraphics{LouisVanLangendonck_AntoninRosa_files/figure-latex/unnamed-chunk-4-1.pdf}

\hypertarget{component-merging}{%
\subsection{4. Component merging}\label{component-merging}}

\begin{Shaded}
\begin{Highlighting}[]
\FunctionTok{library}\NormalTok{(fpc)}
\NormalTok{GMMbic }\OtherTok{\textless{}{-}} \FunctionTok{mclustBIC}\NormalTok{(X,}\AttributeTok{G=}\DecValTok{3}\NormalTok{,}\AttributeTok{modelnames=}\StringTok{\textquotesingle{}VVV\textquotesingle{}}\NormalTok{)}
\NormalTok{GMMmerge }\OtherTok{\textless{}{-}}\NormalTok{ fpc}\SpecialCharTok{::}\FunctionTok{mergenormals}\NormalTok{(X,}\FunctionTok{summary}\NormalTok{(GMMbic,X),}\AttributeTok{method =} \StringTok{"bhat"}\NormalTok{)}
\NormalTok{GMMmerge.ind }\OtherTok{\textless{}{-}}\NormalTok{ GMMmerge}\SpecialCharTok{$}\NormalTok{clustering}
\FunctionTok{plot}\NormalTok{(X,}\AttributeTok{col=}\NormalTok{GMMmerge.ind)}
\end{Highlighting}
\end{Shaded}

\includegraphics{LouisVanLangendonck_AntoninRosa_files/figure-latex/unnamed-chunk-5-1.pdf}

\hypertarget{non-param-density-estimation-of-each-of-the-new-merged-clusters}{%
\subsection{5. Non-param density estimation of each of the new, merged
clusters}\label{non-param-density-estimation-of-each-of-the-new-merged-clusters}}

\begin{Shaded}
\begin{Highlighting}[]
\FunctionTok{plot}\NormalTok{(X,}\AttributeTok{col=}\NormalTok{GMMmerge.ind)}
\ControlFlowTok{for}\NormalTok{ (j }\ControlFlowTok{in} \DecValTok{1}\SpecialCharTok{:}\DecValTok{2}\NormalTok{)\{}
\NormalTok{  cl.j }\OtherTok{\textless{}{-}}\NormalTok{ (GMMmerge.ind}\SpecialCharTok{==}\NormalTok{j)}
  \FunctionTok{sm.density}\NormalTok{(X[cl.j,],}
             \AttributeTok{h=}\FloatTok{0.4}\SpecialCharTok{*}\FunctionTok{c}\NormalTok{(}\FunctionTok{sd}\NormalTok{(X[cl.j,}\StringTok{"temp"}\NormalTok{]),}\FunctionTok{sd}\NormalTok{(X[cl.j,}\StringTok{"casual"}\NormalTok{])),}
             \AttributeTok{display=}\StringTok{"slice"}\NormalTok{,}
             \AttributeTok{props=}\FunctionTok{c}\NormalTok{(}\DecValTok{75}\NormalTok{),}
             \AttributeTok{col=}\NormalTok{j, }
             \AttributeTok{cex=}\DecValTok{4}\NormalTok{, }
             \AttributeTok{add=}\ConstantTok{TRUE}\NormalTok{)}
\NormalTok{\}}
\end{Highlighting}
\end{Shaded}

\includegraphics{LouisVanLangendonck_AntoninRosa_files/figure-latex/unnamed-chunk-6-1.pdf}

\hypertarget{dbscan}{%
\subsection{6. DBScan}\label{dbscan}}

All combinations of parameters are tested, and the average silhouette
across all clusters is computed for each of them. We have chosen the
ideal parameters as those that lead to the highest average silhouette.

\begin{Shaded}
\begin{Highlighting}[]
\FunctionTok{library}\NormalTok{(dbscan)}
\FunctionTok{library}\NormalTok{(cluster)}

\NormalTok{Xs }\OtherTok{\textless{}{-}} \FunctionTok{scale}\NormalTok{(X)}

\NormalTok{eps\_values }\OtherTok{\textless{}{-}} \FunctionTok{c}\NormalTok{(}\FloatTok{0.25}\NormalTok{, }\FloatTok{0.5}\NormalTok{)}
\NormalTok{minPts\_values }\OtherTok{\textless{}{-}} \FunctionTok{c}\NormalTok{(}\DecValTok{10}\NormalTok{, }\DecValTok{15}\NormalTok{, }\DecValTok{20}\NormalTok{)}
\NormalTok{results }\OtherTok{\textless{}{-}} \FunctionTok{data.frame}\NormalTok{()}

\NormalTok{best\_silhouette }\OtherTok{\textless{}{-}} \SpecialCharTok{{-}}\ConstantTok{Inf}
\NormalTok{best\_eps }\OtherTok{\textless{}{-}} \ConstantTok{NULL}
\NormalTok{best\_minPts }\OtherTok{\textless{}{-}} \ConstantTok{NULL}

\ControlFlowTok{for}\NormalTok{ (eps }\ControlFlowTok{in}\NormalTok{ eps\_values) \{}
  \ControlFlowTok{for}\NormalTok{ (minPts }\ControlFlowTok{in}\NormalTok{ minPts\_values) \{}
    \CommentTok{\#Compute DBscan}
\NormalTok{    dbscan\_result }\OtherTok{\textless{}{-}} \FunctionTok{dbscan}\NormalTok{(Xs, }\AttributeTok{eps =}\NormalTok{ eps, }\AttributeTok{minPts =}\NormalTok{ minPts)}
    
    \CommentTok{\#Compute Silhouette,as long as there is more than a single cluster}
    \ControlFlowTok{if}\NormalTok{ (}\FunctionTok{max}\NormalTok{(dbscan\_result}\SpecialCharTok{$}\NormalTok{cluster) }\SpecialCharTok{\textgreater{}} \DecValTok{1}\NormalTok{) \{}
\NormalTok{      silhouette\_avg }\OtherTok{\textless{}{-}} \FunctionTok{silhouette}\NormalTok{(dbscan\_result}\SpecialCharTok{$}\NormalTok{cluster, }\FunctionTok{dist}\NormalTok{(Xs))}
\NormalTok{      avg\_silhouette }\OtherTok{\textless{}{-}} \FunctionTok{mean}\NormalTok{(silhouette\_avg[, }\StringTok{"sil\_width"}\NormalTok{])}
      
\NormalTok{      results }\OtherTok{\textless{}{-}} \FunctionTok{rbind}\NormalTok{(results, }
                       \FunctionTok{data.frame}\NormalTok{(}\AttributeTok{eps =}\NormalTok{ eps, }\AttributeTok{minPts =}\NormalTok{ minPts, }\AttributeTok{silhouette =}\NormalTok{ avg\_silhouette))}
    
      \ControlFlowTok{if}\NormalTok{ (avg\_silhouette }\SpecialCharTok{\textgreater{}}\NormalTok{ best\_silhouette) \{}
\NormalTok{        best\_silhouette }\OtherTok{\textless{}{-}}\NormalTok{ avg\_silhouette}
\NormalTok{        best\_eps }\OtherTok{\textless{}{-}}\NormalTok{ eps}
\NormalTok{        best\_minPts }\OtherTok{\textless{}{-}}\NormalTok{ minPts}
\NormalTok{      \}}
\NormalTok{    \}}
\NormalTok{  \}}
\NormalTok{\}}

\FunctionTok{cat}\NormalTok{(}\StringTok{"Best combination of parameters:}\SpecialCharTok{\textbackslash{}n}\StringTok{"}\NormalTok{)}
\end{Highlighting}
\end{Shaded}

\begin{verbatim}
## Best combination of parameters:
\end{verbatim}

\begin{Shaded}
\begin{Highlighting}[]
\FunctionTok{cat}\NormalTok{(}\StringTok{"Epsilon:"}\NormalTok{, best\_eps, }\StringTok{"}\SpecialCharTok{\textbackslash{}n}\StringTok{"}\NormalTok{)}
\end{Highlighting}
\end{Shaded}

\begin{verbatim}
## Epsilon: 0.5
\end{verbatim}

\begin{Shaded}
\begin{Highlighting}[]
\FunctionTok{cat}\NormalTok{(}\StringTok{"MinPts:"}\NormalTok{, best\_minPts, }\StringTok{"}\SpecialCharTok{\textbackslash{}n}\StringTok{"}\NormalTok{)}
\end{Highlighting}
\end{Shaded}

\begin{verbatim}
## MinPts: 10
\end{verbatim}

\begin{Shaded}
\begin{Highlighting}[]
\NormalTok{dbscan\_result }\OtherTok{\textless{}{-}} \FunctionTok{dbscan}\NormalTok{(Xs, }\AttributeTok{eps =}\NormalTok{ best\_eps, }\AttributeTok{minPts =}\NormalTok{ best\_minPts)}
\FunctionTok{plot}\NormalTok{(Xs, }\AttributeTok{col =}\NormalTok{ dbscan\_result}\SpecialCharTok{$}\NormalTok{cluster, }\AttributeTok{pch =} \DecValTok{19}\NormalTok{, }\AttributeTok{main =} \StringTok{"DBSCAN Clustering"}\NormalTok{)}
\end{Highlighting}
\end{Shaded}

\includegraphics{LouisVanLangendonck_AntoninRosa_files/figure-latex/unnamed-chunk-7-1.pdf}

\begin{Shaded}
\begin{Highlighting}[]
\NormalTok{contingency\_table }\OtherTok{\textless{}{-}} \FunctionTok{table}\NormalTok{(}\AttributeTok{dbscan =}\NormalTok{ dbscan\_result}\SpecialCharTok{$}\NormalTok{cluster, }\AttributeTok{GMM =}\NormalTok{ GMMmerge.ind)}

\FunctionTok{print}\NormalTok{(contingency\_table)}
\end{Highlighting}
\end{Shaded}

\begin{verbatim}
##       GMM
## dbscan   1   2
##      0   1   6
##      1 307   2
##      2   0  50
\end{verbatim}

Even though DBSCAN Cluster 0 (representing noise or unclustered data) is
attributed to two distinct clusters by GMM (GMM Cluster 0 and GMM
Cluster 1), the total number of points in this cluster is relatively low
(7 points in total).

Overall, these results indicate a strong similarity between the
clustering results of DBSCAN and GMM for most observations. Ambiguity is
primarily found in DBSCAN Cluster 0, representing noise or unclustered
data. This similarity may suggest that both methods are uncovering
similar data structures, at least for the data that is meaningfully
clustered.

\hypertarget{interpretation.}{%
\subsection{7. Interpretation.}\label{interpretation.}}

In order to analyze the meaning of the two clustered data points, we
visualized the distributions of every variable conditioned on the
assigned cluster. For categorical features we show the level frequencies
conditioned on the cluster, and for the numerical ones we show the
histograms, also for both clusters.

Looking at the numerical features, the first thing one can notice is
that weather conditions for the second cluster tend to have less
variance and generally be better, so we can confidently hypothesize that
the factors determining the cluster will have something to do with good
weather.

Of course, we see the expected difference in the casual feature as well,
as it also defined the clustering process along with the temperature.
The histograms show how cluster 1 tends to have lower casual values
while cluster 2 has the highest, and also days assigned in cluster 2
have much higher total count of users. Noticing that the mean of
registered users is very similar in days of both clusters while the
number if casual users is significantly larger for cluster 2 days, along
with the better temperatures on those days, we believe we can explain
this behavior by assuming that in cluster 2 days there is increased
demand of bicycles for leisure use. This would also explain why the
number of registered users does not change much, as those use the
service for their work days as well as for leisure days, while the other
users do not have registration and only use it for leisure time, thus
increasing the casual user count on cluster 2 days.

This hypothesis is further reinforced by looking at the categorical
feature distributions depending on the cluster. Just by checking season
and month, one can already assert that cluster 2 days are never during
winter season, which makes sense considering Washington has a cold
climate, making leisure strolls in a bicycle much less appealing.
Finally, the weekday and working day features imply that days assigned
to cluster 2 are weekends, which would fit with our hypothesis of
cluster 2 days being days where users use the service for leisure.

Thus, after analyzing how the features are distributed depending on the
cluster they are assigned to, we believe these clusters indicate two
different kinds of days for the service. * Cluster 1: Work days where
registered users use the service to commute to work. * Cluster 2:
Weekends or days with good temperature where both registered and casual
users use the service for leisure.

\begin{Shaded}
\begin{Highlighting}[]
\NormalTok{df}\OtherTok{\textless{}{-}}\FunctionTok{data.frame}\NormalTok{(day[day}\SpecialCharTok{$}\NormalTok{yr}\SpecialCharTok{==}\DecValTok{1}\NormalTok{,])}
\NormalTok{catCols}\OtherTok{\textless{}{-}}\FunctionTok{c}\NormalTok{(}\DecValTok{3}\NormalTok{,}\DecValTok{5}\NormalTok{,}\DecValTok{6}\NormalTok{,}\DecValTok{7}\NormalTok{,}\DecValTok{8}\NormalTok{,}\DecValTok{9}\NormalTok{)}
\NormalTok{numCols}\OtherTok{\textless{}{-}}\FunctionTok{c}\NormalTok{(}\DecValTok{1}\NormalTok{,}\DecValTok{10}\NormalTok{,}\DecValTok{11}\NormalTok{,}\DecValTok{12}\NormalTok{,}\DecValTok{13}\NormalTok{,}\DecValTok{14}\NormalTok{,}\DecValTok{15}\NormalTok{,}\DecValTok{16}\NormalTok{)}
\NormalTok{numdf}\OtherTok{\textless{}{-}}\NormalTok{df[,numCols]}
\NormalTok{catdf}\OtherTok{\textless{}{-}}\FunctionTok{data.frame}\NormalTok{(}\FunctionTok{lapply}\NormalTok{(df[,catCols],factor))}
\NormalTok{catdf}\SpecialCharTok{$}\NormalTok{cluster}\OtherTok{\textless{}{-}}\FunctionTok{factor}\NormalTok{(dbscan\_result}\SpecialCharTok{$}\NormalTok{cluster)}
\NormalTok{numdf1}\OtherTok{\textless{}{-}}\NormalTok{numdf[catdf}\SpecialCharTok{$}\NormalTok{cluster}\SpecialCharTok{==}\DecValTok{1}\NormalTok{,]}
\NormalTok{numdf2}\OtherTok{\textless{}{-}}\NormalTok{numdf[catdf}\SpecialCharTok{$}\NormalTok{cluster}\SpecialCharTok{==}\DecValTok{2}\NormalTok{,]}
\NormalTok{catdf1}\OtherTok{\textless{}{-}}\NormalTok{catdf[catdf}\SpecialCharTok{$}\NormalTok{cluster}\SpecialCharTok{==}\DecValTok{1}\NormalTok{,]}
\NormalTok{catdf2}\OtherTok{\textless{}{-}}\NormalTok{catdf[catdf}\SpecialCharTok{$}\NormalTok{cluster}\SpecialCharTok{==}\DecValTok{2}\NormalTok{,]}
\FunctionTok{par}\NormalTok{(}\AttributeTok{mfrow =} \FunctionTok{c}\NormalTok{(}\DecValTok{1}\NormalTok{, }\DecValTok{2}\NormalTok{))}
\CommentTok{\#Season}
\FunctionTok{plot}\NormalTok{(catdf1}\SpecialCharTok{$}\NormalTok{season,}\AttributeTok{main=}\StringTok{"Cluster 1: Season"}\NormalTok{)}
\FunctionTok{plot}\NormalTok{(catdf2}\SpecialCharTok{$}\NormalTok{season,}\AttributeTok{main=}\StringTok{"Cluster 2: Season"}\NormalTok{)}
\end{Highlighting}
\end{Shaded}

\includegraphics[height=0.4\textheight]{LouisVanLangendonck_AntoninRosa_files/figure-latex/unnamed-chunk-8-1}

\begin{Shaded}
\begin{Highlighting}[]
\CommentTok{\#Month}
\FunctionTok{plot}\NormalTok{(catdf1}\SpecialCharTok{$}\NormalTok{mnth,}\AttributeTok{main=}\StringTok{"Cluster 1: Month"}\NormalTok{)}
\FunctionTok{plot}\NormalTok{(catdf2}\SpecialCharTok{$}\NormalTok{mnth,}\AttributeTok{main=}\StringTok{"Cluster 2: Month"}\NormalTok{)}
\end{Highlighting}
\end{Shaded}

\includegraphics[height=0.4\textheight]{LouisVanLangendonck_AntoninRosa_files/figure-latex/unnamed-chunk-8-2}

\begin{Shaded}
\begin{Highlighting}[]
\CommentTok{\#Holiday}
\FunctionTok{plot}\NormalTok{(catdf1}\SpecialCharTok{$}\NormalTok{holiday,}\AttributeTok{main=}\StringTok{"Cluster 1: Holiday"}\NormalTok{)}
\FunctionTok{plot}\NormalTok{(catdf2}\SpecialCharTok{$}\NormalTok{holiday,}\AttributeTok{main=}\StringTok{"Cluster 2: Holiday"}\NormalTok{)}
\end{Highlighting}
\end{Shaded}

\includegraphics[height=0.4\textheight]{LouisVanLangendonck_AntoninRosa_files/figure-latex/unnamed-chunk-8-3}

\begin{Shaded}
\begin{Highlighting}[]
\CommentTok{\#weekday}
\FunctionTok{plot}\NormalTok{(catdf1}\SpecialCharTok{$}\NormalTok{weekday,}\AttributeTok{main=}\StringTok{"Cluster 1: Weekday"}\NormalTok{)}
\FunctionTok{plot}\NormalTok{(catdf2}\SpecialCharTok{$}\NormalTok{weekday,}\AttributeTok{main=}\StringTok{"Cluster 2: Weekday"}\NormalTok{)}
\end{Highlighting}
\end{Shaded}

\includegraphics[height=0.4\textheight]{LouisVanLangendonck_AntoninRosa_files/figure-latex/unnamed-chunk-8-4}

\begin{Shaded}
\begin{Highlighting}[]
\CommentTok{\#workingday}
\FunctionTok{plot}\NormalTok{(catdf1}\SpecialCharTok{$}\NormalTok{workingday,}\AttributeTok{main=}\StringTok{"Cluster 1: Workingday"}\NormalTok{)}
\FunctionTok{plot}\NormalTok{(catdf2}\SpecialCharTok{$}\NormalTok{workingday,}\AttributeTok{main=}\StringTok{"Cluster 2: Workingday"}\NormalTok{)}
\end{Highlighting}
\end{Shaded}

\includegraphics[height=0.4\textheight]{LouisVanLangendonck_AntoninRosa_files/figure-latex/unnamed-chunk-8-5}

\begin{Shaded}
\begin{Highlighting}[]
\CommentTok{\#Weathersit}
\FunctionTok{plot}\NormalTok{(catdf1}\SpecialCharTok{$}\NormalTok{weathersit,}\AttributeTok{main=}\StringTok{"Cluster 1: Weathersit"}\NormalTok{)}
\FunctionTok{plot}\NormalTok{(catdf2}\SpecialCharTok{$}\NormalTok{weathersit,}\AttributeTok{main=}\StringTok{"Cluster 2: Weathersit"}\NormalTok{)}
\end{Highlighting}
\end{Shaded}

\includegraphics[height=0.4\textheight]{LouisVanLangendonck_AntoninRosa_files/figure-latex/unnamed-chunk-8-6}

\begin{Shaded}
\begin{Highlighting}[]
\CommentTok{\#Temperature}
\FunctionTok{hist}\NormalTok{(numdf1}\SpecialCharTok{$}\NormalTok{temp,}\AttributeTok{main=}\StringTok{"Cluster 1: Temperature"}\NormalTok{,}\AttributeTok{xlim =} \FunctionTok{c}\NormalTok{(}\DecValTok{0}\NormalTok{,}\DecValTok{1}\NormalTok{))}
\FunctionTok{hist}\NormalTok{(numdf2}\SpecialCharTok{$}\NormalTok{temp,}\AttributeTok{main=}\StringTok{"Cluster 2: Temperature"}\NormalTok{,}\AttributeTok{xlim =} \FunctionTok{c}\NormalTok{(}\DecValTok{0}\NormalTok{,}\DecValTok{1}\NormalTok{))}
\end{Highlighting}
\end{Shaded}

\includegraphics[height=0.4\textheight]{LouisVanLangendonck_AntoninRosa_files/figure-latex/unnamed-chunk-8-7}

\begin{Shaded}
\begin{Highlighting}[]
\CommentTok{\#Feeling temperature}
\FunctionTok{hist}\NormalTok{(numdf1}\SpecialCharTok{$}\NormalTok{atemp,}\AttributeTok{main=}\StringTok{"Cluster 1: Feeling temperature"}\NormalTok{,}\AttributeTok{xlim =} \FunctionTok{c}\NormalTok{(}\DecValTok{0}\NormalTok{,}\DecValTok{1}\NormalTok{))}
\FunctionTok{hist}\NormalTok{(numdf2}\SpecialCharTok{$}\NormalTok{atemp,}\AttributeTok{main=}\StringTok{"Cluster 2: Feeling temperature"}\NormalTok{,}\AttributeTok{xlim =} \FunctionTok{c}\NormalTok{(}\DecValTok{0}\NormalTok{,}\DecValTok{1}\NormalTok{))}
\end{Highlighting}
\end{Shaded}

\includegraphics[height=0.4\textheight]{LouisVanLangendonck_AntoninRosa_files/figure-latex/unnamed-chunk-8-8}

\begin{Shaded}
\begin{Highlighting}[]
\CommentTok{\#Humidity}
\FunctionTok{hist}\NormalTok{(numdf1}\SpecialCharTok{$}\NormalTok{hum,}\AttributeTok{main=}\StringTok{"Cluster 1: Humidity"}\NormalTok{,}\AttributeTok{xlim =} \FunctionTok{c}\NormalTok{(}\DecValTok{0}\NormalTok{,}\DecValTok{1}\NormalTok{))}
\FunctionTok{hist}\NormalTok{(numdf2}\SpecialCharTok{$}\NormalTok{hum,}\AttributeTok{main=}\StringTok{"Cluster 2: Humidity"}\NormalTok{,}\AttributeTok{xlim =} \FunctionTok{c}\NormalTok{(}\DecValTok{0}\NormalTok{,}\DecValTok{1}\NormalTok{))}
\end{Highlighting}
\end{Shaded}

\includegraphics[height=0.4\textheight]{LouisVanLangendonck_AntoninRosa_files/figure-latex/unnamed-chunk-8-9}

\begin{Shaded}
\begin{Highlighting}[]
\CommentTok{\#Wind Speed}
\FunctionTok{hist}\NormalTok{(numdf1}\SpecialCharTok{$}\NormalTok{windspeed,}\AttributeTok{main=}\StringTok{"Cluster 1: Windspeed"}\NormalTok{,}\AttributeTok{xlim =} \FunctionTok{c}\NormalTok{(}\DecValTok{0}\NormalTok{,}\FloatTok{0.55}\NormalTok{))}
\FunctionTok{hist}\NormalTok{(numdf2}\SpecialCharTok{$}\NormalTok{windspeed,}\AttributeTok{main=}\StringTok{"Cluster 2: Windspeed"}\NormalTok{,}\AttributeTok{xlim =} \FunctionTok{c}\NormalTok{(}\DecValTok{0}\NormalTok{,}\FloatTok{0.55}\NormalTok{))}
\end{Highlighting}
\end{Shaded}

\includegraphics[height=0.4\textheight]{LouisVanLangendonck_AntoninRosa_files/figure-latex/unnamed-chunk-8-10}

\begin{Shaded}
\begin{Highlighting}[]
\CommentTok{\#Casual}
\FunctionTok{hist}\NormalTok{(numdf1}\SpecialCharTok{$}\NormalTok{casual,}\AttributeTok{main=}\StringTok{"Cluster 1: Casual"}\NormalTok{,}\AttributeTok{xlim =} \FunctionTok{c}\NormalTok{(}\DecValTok{0}\NormalTok{,}\DecValTok{3300}\NormalTok{))}
\FunctionTok{hist}\NormalTok{(numdf2}\SpecialCharTok{$}\NormalTok{casual,}\AttributeTok{main=}\StringTok{"Cluster 2: Casual"}\NormalTok{,}\AttributeTok{xlim =} \FunctionTok{c}\NormalTok{(}\DecValTok{0}\NormalTok{,}\DecValTok{3300}\NormalTok{))}
\end{Highlighting}
\end{Shaded}

\includegraphics[height=0.4\textheight]{LouisVanLangendonck_AntoninRosa_files/figure-latex/unnamed-chunk-8-11}

\begin{Shaded}
\begin{Highlighting}[]
\CommentTok{\#Registered}
\FunctionTok{hist}\NormalTok{(numdf1}\SpecialCharTok{$}\NormalTok{registered,}\AttributeTok{main=}\StringTok{"Cluster 1: Registered"}\NormalTok{,}\AttributeTok{xlim =} \FunctionTok{c}\NormalTok{(}\DecValTok{0}\NormalTok{,}\DecValTok{7000}\NormalTok{))}
\FunctionTok{hist}\NormalTok{(numdf2}\SpecialCharTok{$}\NormalTok{registered,}\AttributeTok{main=}\StringTok{"Cluster 2: Registered"}\NormalTok{,}\AttributeTok{xlim =} \FunctionTok{c}\NormalTok{(}\DecValTok{0}\NormalTok{,}\DecValTok{7000}\NormalTok{))}
\end{Highlighting}
\end{Shaded}

\includegraphics[height=0.4\textheight]{LouisVanLangendonck_AntoninRosa_files/figure-latex/unnamed-chunk-8-12}

\begin{Shaded}
\begin{Highlighting}[]
\CommentTok{\#Count}
\FunctionTok{hist}\NormalTok{(numdf1}\SpecialCharTok{$}\NormalTok{cnt,}\AttributeTok{main=}\StringTok{"Cluster 1: Count"}\NormalTok{,}\AttributeTok{xlim =} \FunctionTok{c}\NormalTok{(}\DecValTok{0}\NormalTok{,}\DecValTok{10000}\NormalTok{))}
\FunctionTok{hist}\NormalTok{(numdf2}\SpecialCharTok{$}\NormalTok{cnt,}\AttributeTok{main=}\StringTok{"Cluster 2: Count"}\NormalTok{,}\AttributeTok{xlim =} \FunctionTok{c}\NormalTok{(}\DecValTok{0}\NormalTok{,}\DecValTok{10000}\NormalTok{))}
\end{Highlighting}
\end{Shaded}

\includegraphics[height=0.4\textheight]{LouisVanLangendonck_AntoninRosa_files/figure-latex/unnamed-chunk-8-13}

\end{document}
